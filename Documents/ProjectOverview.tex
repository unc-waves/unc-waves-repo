\documentclass[12pt]{article}

\usepackage[margin=1in]{geometry} 
\usepackage{amsmath,amsthm,amssymb}
\usepackage[pdftex]{graphicx}

\newcommand{\N}{\mathbb{N}}
\newcommand{\Z}{\mathbb{Z}}

\newenvironment{theorem}[2][Theorem]{\begin{trivlist}
\item[\hskip \labelsep {\bfseries #1}\hskip \labelsep {\bfseries #2.}]}{\end{trivlist}}
\newenvironment{lemma}[2][Lemma]{\begin{trivlist}
\item[\hskip \labelsep {\bfseries #1}\hskip \labelsep {\bfseries #2.}]}{\end{trivlist}}
\newenvironment{exercise}[2][Exercise]{\begin{trivlist}
\item[\hskip \labelsep {\bfseries #1}\hskip \labelsep {\bfseries #2.}]}{\end{trivlist}}
\newenvironment{problem}[2][Problem]{\begin{trivlist}
\item[\hskip \labelsep {\bfseries #1}\hskip \labelsep {\bfseries #2.}]}{\end{trivlist}}
\newenvironment{question}[2][Question]{\begin{trivlist}
\item[\hskip \labelsep {\bfseries #1}\hskip \labelsep {\bfseries #2.}]}{\end{trivlist}}
\newenvironment{corollary}[2][Corollary]{\begin{trivlist}
\item[\hskip \labelsep {\bfseries #1}\hskip \labelsep {\bfseries #2.}]}{\end{trivlist}}

\begin{document}

% --------------------------------------------------------------
%                         Start here
% --------------------------------------------------------------

\title{Making Waves}%replace X with the appropriate number
\author{Chief Project Manager: Alex French \\ 
Chief Editor: Steven Love\\
Chief Architect: Renato Pereyra\\
Chief Client Manager: Christopher Mullins\\ %replace with your name
COMP523 Software Development - Stotts} %if necessary, replace with your course title

\maketitle

\section{Concept}

We will build an interface between a waveform generator and streaming digital outputs from multiple sound cards. Up to sixteen parallel channels are needed to maintain synchronization across all channels.

\subsubsection{Expected Project Parameters}
\begin{itemize}
\item Resample to sync uniform times
\begin{itemize}
	\item Wave likely provided as 3-d matrix (x, y, t) with value of excitement at that time and place.
	\item Wave data provided by mathematicians may need to be interpolated in order to produce a wave that works with the hardware. (Example: Mathematician gives wave at 10ms or 200ms time intervals, and we need 100ms intervals).
	\item Possible warnings generated when interpolation differs over a threshold from the original wave?

\end{itemize}
\item Package in data record with metadata
\item (Optional) Store resultant data in a file
\item Translate to LewOS Macro
\item LewOS Simulator
\item Physics Simulator
\item Graphic result display
\end{itemize}

\subsubsection{Possible Extensions}

\begin{itemize}
\item MATLAB plugin
\begin{itemize}
	\item Mathematicians may use MATLAB to generate a  waveform and then provide corresponding matrix to our software. An optional extension might include a MATLAB plugin to expedite this process.
\end{itemize}
\end{itemize}

\begin{center}
\includegraphics[scale = 0.5]{UncWaveConcept.png}
\end{center}

NB. We will have a revised version of this graphic soon.
\end{document}
