
\documentclass[12pt]{article}

\usepackage[margin=1in]{geometry} 
\usepackage{amsmath,amsthm,amssymb}
\usepackage[pdftex]{graphicx}

\newcommand{\N}{\mathbb{N}}
\newcommand{\Z}{\mathbb{Z}}

\newenvironment{theorem}[2][Theorem]{\begin{trivlist}
\item[\hskip \labelsep {\bfseries #1}\hskip \labelsep {\bfseries #2.}]}{\end{trivlist}}
\newenvironment{lemma}[2][Lemma]{\begin{trivlist}
\item[\hskip \labelsep {\bfseries #1}\hskip \labelsep {\bfseries #2.}]}{\end{trivlist}}
\newenvironment{exercise}[2][Exercise]{\begin{trivlist}
\item[\hskip \labelsep {\bfseries #1}\hskip \labelsep {\bfseries #2.}]}{\end{trivlist}}
\newenvironment{problem}[2][Problem]{\begin{trivlist}
\item[\hskip \labelsep {\bfseries #1}\hskip \labelsep {\bfseries #2.}]}{\end{trivlist}}
\newenvironment{question}[2][Question]{\begin{trivlist}
\item[\hskip \labelsep {\bfseries #1}\hskip \labelsep {\bfseries #2.}]}{\end{trivlist}}
\newenvironment{corollary}[2][Corollary]{\begin{trivlist}
\item[\hskip \labelsep {\bfseries #1}\hskip \labelsep {\bfseries #2.}]}{\end{trivlist}}

\begin{document}

% --------------------------------------------------------------
%                         Start here
% --------------------------------------------------------------

\title{Making Waves: Project Specifications}%replace X with the appropriate number
\author{Chief Project Manager: Alex French \\ 
Chief Editor: Steven Love\\
Chief Architect: Renato Pereyra\\
Chief Client Manager: Christopher Mullins\\ %replace with your name
COMP523 Software Development - Stotts} %if necessary, replace with your course title

\maketitle

\section{Usability}  %PErformance, Supportability, SEcurity, Interface, Assumptions


\begin{enumerate}
\item Users are assumed to be Applied Mathematicians familiar with wave terminology which may be used in any messages generated by the application, including messages to specify input or to notify of improper input format.
\item For ease of use, input format will be specified once upon start of program application, repeated upon input error, and accessible in written documentation.
\item Because user interaction is simple, UI should be intuitive.
\item Limiting factors on speed of task completion are interpolation and generation of graphic representation. Efficiency of graphic representation is less important, since this will be removed for eventual use of application (live wave tank). Efficiency of interpolation depends on interpolation techniques. We may be able to time-limit these calculations, with an option to continue or terminate after time-limit reached.
\end{enumerate}

\section{Performance}

\begin{enumerate}
\item The interpolation mechanism implemented must guard against foreseeable misbehaviors resulting from round-off error.  In addition, the mechanism must be able to interpolate large datasets within a reasonable amount of time.
\item Matlab input that is outside the operating range of the hardware must be identified prior to execution by the hardware.  This may be accomplished by utilizing specialized constraints when performing a LewOS simulation of the output.
\item Prior to performing any other operations, the Matlab plug-in must not corrupt any open Matlab programs.
\item The Matlab plug-in built must not cause Matlab to crash for any reason. To achieve this, the input must be thoroughly checked to ensure LewOS input can be generated. If valid LewOS instructions cannot be generated, the plug-in must terminate, generate an appropriate error message, and return control to Matlab/user.
\item The system built must be largely self-contained. That is, it must not rely on any user-specific packages. Ideally, the system will be easily ported to any machine which meets some minimum set of requirements, including, but not limited to, Matlab, a C/C++ compiler, and the LewOS architecture/simulator.
\end{enumerate}


\section{Supportability}

\begin{enumerate}
\item External written documentation shall be provided which includes information pertinent to users (input format, wave limitations like wave height and detail, interpolation math, etc.) and information pertinent to those maintaining code (structure, etc.).
\item A brief version of the user-related information may be accessed through the 'help' command (or similar).
\item Required training should be minimal, as user interaction is generally limited to input in strict format.
\item Code itself shall be commented to describe the purpose and use of obscure or lengthy methods or blocks of code.
\end{enumerate}


\section{Security}

\begin{enumerate}
\item No log-in functionality or user authentication provided.
\item No audits necessary.
\end{enumerate}

\section{Interface}

\begin{enumerate}
\item The interface for the UNC Waves Team project will take the form of a MATLAB plugin.  A researcher will use MATLAB to generate a 3-dimensional matrix representing a waveform they hope to reproduce in the tank.  This way the researcher can provide the waveform to our software without leaving the MATLAB environment. 
\item The interface will be robust and loosely bound to the MATLAB environment.  In order to ensure this, our plugin will be able to identify malformed input, which includes input of the wrong data type, or input of numerical matrices of the wrong dimension.  There will be appropriate error messages regarding all of these situations.  As soon as the input is verified, the plugin will perform any necessary interpolation on the data, and provide instructions to the LewOS system generated from these data, to be saved to a file and read from LewOS.
\item Additionally, the plugin will contain a LewOS simulator, which displays a visual simulation of the excitation patterns which will be triggered in the tank through LewOS, according to the input of the user.
\end{enumerate}

\section{Assumptions}

\begin{enumerate}
\item There will be 16 or fewer exciters.
\item Our software will ignore hardware issues.
\item The input is constrained to be a non-empty 3-dimensional MATLAB array representing excitation patterns of a wave.
	\begin{itemize}
	\item If the input is malformed or cannot be interpolated in order to derive the waveform, then our software should give an error message and not continue.
	\end{itemize}
\end{enumerate}




\end{document}



